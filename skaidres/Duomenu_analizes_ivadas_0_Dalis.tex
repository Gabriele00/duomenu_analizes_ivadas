\documentclass[11pt,xcolor=table]{beamer}
\usetheme{CambridgeUS}
\usecolortheme{dolphin}

\usepackage[utf8]{inputenc}
\usepackage[L7x]{fontenc}
\usepackage[lithuanian]{babel}
\usepackage{lmodern} %be neveikia bold, italic etc


\usepackage{amsmath}
\usepackage{amsfonts}
\usepackage{amssymb}

\usepackage{graphicx}
\graphicspath{{./figures/}}

\usepackage{adjustbox}
\usepackage{bm}
\usepackage{subcaption}

\usepackage{multirow} % for multirow tables
\setbeamertemplate{caption}[numbered] % for numbering captions

\usepackage{forest}


\usepackage{listings}
\usepackage{color}
\usepackage{textcomp}
\definecolor{listinggray}{gray}{0.9}
\definecolor{lbcolor}{rgb}{0.9,0.9,0.9}
\lstset{
	backgroundcolor=\color{lbcolor},
	tabsize=4,
	rulecolor=,
	language=matlab,
        basicstyle=\scriptsize,
        upquote=true,
        aboveskip={0.5\baselineskip},
        belowskip={0.5\baselineskip},
        columns=fixed,
        showstringspaces=false,
        extendedchars=false,
        breaklines=true,
        prebreak = \raisebox{0ex}[0ex][0ex]{\ensuremath{\hookleftarrow}},
        frame=single,
        showtabs=false,
        showspaces=false,
        showstringspaces=false,
        identifierstyle=\ttfamily,
        keywordstyle=\color[rgb]{0,0,1},
        commentstyle=\color[rgb]{0.133,0.545,0.133},
        stringstyle=\color[rgb]{0.627,0.126,0.941}
}


\usepackage{tikz, pgfplots}
\usetikzlibrary{patterns,decorations.pathreplacing}


\author{Justas Mundeikis}
\title{Duomenų analizės įvadas}
\subtitle{Intro}
%\setbeamercovered{transparent} 
%\setbeamertemplate{navigation symbols}{} 
%\logo{} 
%\institute{} 
%\date{} 
%\subject{} 



%----------------------------------------------------------
\begin{document}
%----------------------------------------------------------
\section{Apie kursą}
%----------------------------------------------------------

\begin{frame}
\titlepage
\end{frame}


\begin{frame}
\Large In God we trust, all others bring data\\ 
\scriptsize - W. Edwards Deming

\end{frame}

%----------------------------------------------------------
\begin{frame}
\begin{table}[]
\caption{Shared 'Dublin' descriptors for Short Cycle, First Cycle, Second Cycle and Third Cycle Awards, 2004}
\label{shared_dublin_2004}
\scalebox{0.45}{
\begin{tabular}{|l|l|l|l|}
\hline
 & \textbf{Bachelor} & \textbf{Master} & \textbf{Doctorate} \\ \hline
Knowledge and understanding: & \begin{tabular}[c]{@{}l@{}}{[}Is{]} supported by advanced text \\ books {[}with{]} some aspects \\ informed by knowledge at the \\ forefront of their field of study ..\end{tabular} & \begin{tabular}[c]{@{}l@{}}provides a basis or \\ opportunity for originality \\ in developing or applying \\ ideas often in \\ a research* context ..\end{tabular} & \begin{tabular}[c]{@{}l@{}}{[}includes{]} a systematic understanding of \\ their field of study and mastery of the \\ methods of research* associated \\ with that field..\end{tabular} \\ \hline
\begin{tabular}[c]{@{}l@{}}Applying knowledge \\ and understanding:\end{tabular} & \begin{tabular}[c]{@{}l@{}}{[}through{]} devising and \\ sustaining arguments\end{tabular} & \begin{tabular}[c]{@{}l@{}}{[}through{]} problem solving \\ abilities {[}applied{]} in new \\ or unfamiliar environments\\ within broader \\ (or multidisciplinary) contexts ..\end{tabular} & \begin{tabular}[c]{@{}l@{}}{[}is demonstrated by the{]} ability to conceive,\\ design, implement and adapt a substantial \\ process of research* with scholarly integrity ..\\ {[}is in the context of{]} a contribution that \\ extends the frontier of knowledge by \\ developing a substantial body of work some \\ of which merits national or international \\ refereed publication .\end{tabular} \\ \hline
\begin{tabular}[c]{@{}l@{}}Making \\ judgements:\end{tabular} & \begin{tabular}[c]{@{}l@{}}{[}involves{]} gathering and \\ interpreting relevant data ..\end{tabular} & \begin{tabular}[c]{@{}l@{}}{[}demonstrates{]} the ability to \\ integrate knowledge and \\ handle complexity, and \\ formulate judgements \\ with incomplete data ..\end{tabular} & \begin{tabular}[c]{@{}l@{}}{[}requires being{]} capable of \\ critical analysis, evaluation and \\ synthesis of new and complex ideas..\end{tabular} \\ \hline
Communication & \begin{tabular}[c]{@{}l@{}}{[}of{]} information, ideas,\\ problems and solutions ..\end{tabular} & \begin{tabular}[c]{@{}l@{}}{[}of{]} their conclusions and the \\ underpinning knowledge and \\ rationale (restricted scope) \\ to specialist and non-specialist \\ audiences (monologue) ..\end{tabular} & \begin{tabular}[c]{@{}l@{}}with their peers, the larger scholarly\\ community and with society in \\ general (dialogue) about their \\ areas of expertise (broad scope)..\end{tabular} \\ \hline
Learning skills .. & \begin{tabular}[c]{@{}l@{}}have developed those \\ skills needed to study \\ further with a high \\ level of autonomy ..\end{tabular} & \begin{tabular}[c]{@{}l@{}}study in a manner that may\\ be largely self-directed \\ or autonomous..\end{tabular} & \begin{tabular}[c]{@{}l@{}}expected to be able to promote, \\ within academic and professional \\ contexts, technological, \\ social or cultural advancement ..\end{tabular} \\ \hline
\end{tabular}
}
\end{table}
\end{frame}


\begin{frame}{Kurso tikslas}
\begin{itemize}
\item Suteikti pagrindines reikalingas kompetencijas darbui su duomenimis
\item Ugdyti studentų gebėjimą savarankiškai rinkti ir analizuoti mokslinę literatūrą bei duomenis, pateikti apibendrintas įžvalgas
\item Kursas yra "lengvas", tačiau reikalauja daug praktinio darbo pastangų
\item Dirbsime tik su R, RStudio, Git ir Github (Jupyter?)
\item Neliesime Matlab, Octave, Python, Julia, Eviews
\end{itemize}
\end{frame}


%----------------------------------------------------------

\begin{frame}{Kurso turinys}
\begin{enumerate}
\item Command-line interface,  Git ir GitHub, Google Scholar
\item R ir RStudio, R programavimo pagrindai
\item Tiriamoji duomenų analizė (angl.: exploratory data analysis)
\item Atkartojami tyrimai (angl.: reproducible research) su LaTex, RMarkdown
\end{enumerate}
\end{frame}

%----------------------------------------------------------
\begin{frame}{Sando pristatymas}
\begin{itemize}
\item Kurso apimtis 130 akad. val.
\begin{itemize}
\item Paskaitos 32 akad. val.
\item Seminarai 16 akad. val.
\item Savarankiškas darbas 82 akad. val.
\end{itemize}
\item 6 akad. val. per 2 savaites
\item 10.5 akad. val. savarankiško darbo per dvi savaites
\end{itemize}
\end{frame}

%----------------------------------------------------------
\begin{frame}{Studijos forma}
\begin{itemize}
\item Paskaitos
\begin{itemize}
\item Paskaitos vyks prie PC
\item Idealiu atveju po 2 paskaitų, trečia paskaita bus seminaras, kai studentai savarankiškai bandys spręsti uždavinius
\item Taip pat paskaitų ir seminarų metu bus aptariamos namų darbų užduotys
\item Paskaitos ir seminarai nėra privalomi, bet rekomenduotini
\end{itemize}
\item Savarankiškas darbas
\begin{itemize}
\item Namų darbų ruošimas (laiko langas uždaromas 2 val. prieš paskaitą), du bandymai su 6 val. pertrauka
\item Rašto darbo paruošimas (esė)
\item Kitų ($\approx 2-3$) studentų rašto darbų vertinimas
\item Savarankiškas pateiktos literatūros studijavimas
\item Žinių skaitymas
\end{itemize}
\end{itemize}
\end{frame}

%----------------------------------------------------------

\begin{frame}
\begin{itemize}
\item Visa studijų medžiaga \href{https://github.com/justasmundeikis/duomenu_analizes_ivadas}{\color{blue}{Github}}, tik pirmos paskaitos medžiaga @VMA
\item Kontaktai: justas.mundeikis@evaf.vu.lt
\begin{itemize}
\item Subject: "data-science"
\item Atsakau įprastai per 3 dd.
\end{itemize}
\end{itemize}
\end{frame}

%----------------------------------------------------------
\begin{frame}{Kurso vertinimo strategija}

\begin{itemize}
\item Namų darbai - 20\%
\\Namų darbų tikslas parodyti studentams, kokio pobūdžio uždavinius studentai turi gebėti savarankiškai spręsti. Namų darbų vertinimas: maksimumas iš leidžiamų 2 bandymų, arba grupinis vertinimas
\item Neanonsuoti testai - 20\%
\\Neanonsuoti trumpi testai skirti užtikrinti, jog studentai nuolatos skirtų deramą dėmesį ir laiku įdėtų reikalingas pastangas studijoms
\item Savarankiškas darbas (esė) - 20\%
\item Baigiamasis egzaminas - 40\% 
\\Egzamine tikrinamos tiek teorijos tiek įgytos praktinės žinios 
\\Egzaminas turi būti išlaikytas ne mažiau 50\%, jeigu rezultatas <50\% egzmainas turi būti perlaikytas
\item $BP=mean(0.2*mean(ND)+0.2*mean(NT)+0.2*SD +0.4*BE)$
\end{itemize}
\end{frame}
%--------------

\begin{frame}{Tvarka egzamino metu}
\begin{itemize}
\item Studentas neturi naudotis jokiais šaltiniais ir
priemonėmis, kad nesukeltų įtarimų dėl savo
nesąžiningumo studijų rezultatų vertinimo metu.
\item  Neleistinų šaltinių ir priemonių turėjimas, interneto
tinklalapių atsidarymas pripažįstamas pakankamu
įrodymu, kad studentas šiomis priemonėmis
naudojosi.
\item  Identifikavus, kad studentas naudojasi neleistinais
šaltiniais ar priemonėmis, jam toliau egzamino laikyti
neleidžiama, o informacija apie įvykį perduodama
fakulteto administracijai.
\item Studentai / stebėtojai prižiūrės tvarką egzamino
metu, bus įrašomas video ekranų turinys
\item Konkrečiai kas galima, kas ne, nuspręsime semestro pabaigoje
\end{itemize}
\end{frame}

\begin{frame}{Esė darbas (bus papildyta...!) }
\begin{itemize}
\item Tema iki 2019-04-01 (sugalvoti savarankiškai)
\item Planas iki 2019-05-01
\item Įkėlimas iki 2019-06-05
\item Apimtis: >=3000 žodžių (10-15 psl)
\item Kalba: LT/EN
\item Vertinimo kritierijai:
\begin{itemize}
\item Esė turinio ir temos pavadinimo atitikimas
\item struktūros aiškumas
\item loginė dėstymo seka, gebėjimas sieti mintis
\item argumentų tinkamumas, teiginių logika ir aiškumas;
\item originalumas;
\item išvadų pagrįstumas
\item citavimo teisingumas ir įforminimas
\item veikiantis R kodas
\item Mažiausiai 10 mokslinių šaltinių finaliniame variante
\end{itemize}
\end{itemize}
\end{frame}

\begin{frame}{Esė planas}
Esė darbo planą turėtų sudaryti:
\begin{itemize}
\item Įvadas, kuriame nurodomas esė tikslas (pagrindinis klausimas), tarpiniai uždaviniai, esė objektas
\item 3-5 pagrindinės dalys
\item Esė struktūra, kuri turi turėti aiškią sistemą ir tos sistemos logiką
\item Trumpai aprašyta kokie klausimai bus aptariami kiekvienoje dalyje, kaip ir kodėl.
\item Literatūros sąrašas, naudotas planui parengti mažiausiai 5 moksliniai šaltiniai
\end{itemize}
\end{frame}


\begin{frame}{Literatūra}
\begin{itemize}
\item Nurodyta sande
\item Tačiau praktiškai, kiekvieno skaidrių set'o pabaigoje, bus pateikti rekomenduotini šaltiniai
\end{itemize}
\end{frame}


\begin{frame}{Pastaba}
\begin{itemize}
\item Paskaitos yra ruošiamos "as we go"
\item Todėl galimos klaidos pirminėse skaidrėse
\item Esminės klaidos bus pataisytos tą pačią dieną (pvz. neveikiantis kodas)
\item Mažiau svarbios klaidos tikėtina irgi, visgi "final version" tą dieną, kai pradedama nauja tema
\item Atsižvelgiant į "as we go", studentai kviečiami teikti savo siūlymus, pastabas ir tiesiogiai daryti įtaką kurso turiniui
\end{itemize}
\end{frame}












\end{document}