\documentclass[11pt,xcolor=table]{beamer}
\usetheme{CambridgeUS}
\usecolortheme{dolphin}

\usepackage[utf8]{inputenc}
\usepackage[L7x]{fontenc}
\usepackage[lithuanian]{babel}
\usepackage{lmodern} %be neveikia bold, italic etc


\usepackage{amsmath}
\usepackage{amsfonts}
\usepackage{amssymb}

\usepackage{graphicx}
\graphicspath{{./figures/}}

\usepackage{adjustbox}
\usepackage{bm}
\usepackage{subcaption}

\usepackage{multirow} % for multirow tables
\setbeamertemplate{caption}[numbered] % for numbering captions

\usepackage{forest}


\usepackage{listings}
\usepackage{color}
\usepackage{textcomp}
\definecolor{listinggray}{gray}{0.9}
\definecolor{lbcolor}{rgb}{0.9,0.9,0.9}
\lstset{
	backgroundcolor=\color{lbcolor},
	tabsize=4,
	rulecolor=,
	language=r,
        basicstyle=\scriptsize,
        upquote=true,
        aboveskip={0.5\baselineskip},
        belowskip={0.5\baselineskip},
        columns=fixed,
        showstringspaces=false,
        extendedchars=false,
        breaklines=true,
        prebreak = \raisebox{0ex}[0ex][0ex]{\ensuremath{\hookleftarrow}},
        frame=single,
        showtabs=false,
        showspaces=false,
        showstringspaces=false,
        identifierstyle=\ttfamily,
        keywordstyle=\color[rgb]{0,0,1},
        commentstyle=\color[rgb]{0.133,0.545,0.133},
        stringstyle=\color[rgb]{0.627,0.126,0.941}
}


\usepackage{tikz, pgfplots}
\usetikzlibrary{patterns,decorations.pathreplacing}


\author{Justas Mundeikis}
\title{Duomenų analizės įvadas}
\subtitle{2-1 Dalis}
%\setbeamercovered{transparent} 
%\setbeamertemplate{navigation symbols}{} 
%\logo{} 
%\institute{Vilniaus Universitetas, EVAF} 
%\date{} 
%\subject{} 


\setbeamerfont{subsection in toc}{size=\scriptsize}
%----------------------------------------------------------
\begin{document}
%----------------------------------------------------------

\begin{frame}
\titlepage
\end{frame}

%----------------------------------------------------------

\begin{frame}{1. Dalies turinys}
\tableofcontents
\end{frame}

%----------------------------------------------------------
\section{R programavimas}
%----------------------------------------------------------
%----------------------------------------------------------
\subsection{R Istorija}
%---------------------------------------------------------

\begin{frame}{R Istorija}
\begin{itemize}
\item R yra S kalbos dialektas
\item S kalba parašyta John Chambers et al. @Bell Labs 1976m.
\item S buvo perrašyta 1988m. (v3) ir tapo labiau panaši į statistinę programą, o 1998m. išleista v4.
\item R sukurtas 1991m. mokslinio darbo rėmuose (Ross Ihaka ir Robert Gentleman ) Auckland universitete (Naujoji Zealandija)
\item 1993m. R pristatytas visuomenei
\item 1995m. R gavo GNU licenziją
\end{itemize}
\end{frame}

%---------------------------------------------------------

\begin{frame}{R Istorija}
\begin{itemize}
\item R sintaksė labai panaši į S 
\item R veikia ant su bet kokia operacine sistema
\item Didelė bendruomenė, todėl labai daug paketų ir dažni bugfix'ai
\item Santykinai lengva atlikti statistines analizes, tačiau suteikia beveik neribotas galimybes norintiems programuoti savo paketus
\item R yra laisva programa (\textit{free software}) remiantis GNU Public License
\end{itemize}
\end{frame}

%---------------------------------------------------------

\begin{frame}{Free software}
Jeigu kalbame apie "free software" turima omenyje 4 laisves
\begin{itemize}
\item 0. Laisvė naudotis programa, bet kuriuo tikslu
\item 1. Laisvė analizuoti ir keisti programą pagal savo poreikius
\item 2. Laisvė dalintis programos kopijomis
\item 3. Laisvė dalintis pagerintomis kopijomis
\end{itemize}
1 ir 3 laisvėms būtina laisva prieiga prie programos kodo.\\
\href{https://www.gnu.org/philosophy/philosophy.html}{\color{blue}{Philosophy of the GNU Project}}
\end{frame}

%---------------------------------------------------------

\begin{frame}{R minusai}
\begin{itemize}
\item R remiasi 40 metų senumo programa, todėl trūksta 3D grafikų 
\item Paketus kuria patys vartotojai, todėl jeigu nėra jau sukurto reikiamo funkcionalumo, reikia kurti pačiam
\item Visi objektai R turi būti įkeliami į darbinę atmintį
\item R nėra labai universali kalba
\end{itemize}
\end{frame}

%---------------------------------------------------------

\begin{frame}{R ir RStudio instaliavimas}
\begin{itemize}
\item R reikia instaliuoti iš CRAN
\item \href{https://cran.r-project.org/}{\textcolor{blue}{https://cran.r-project.org/}}
\item Paleidžiame R 
\item Tam kad būtų lengviau dirbti su R, turėti aibę papildomų funkcijų, instaliuojame RStudio
\item \href{https://www.rstudio.com/products/rstudio/download/}{\textcolor{blue}{https://www.rstudio.com/products/rstudio/download/}}
\item Startuojame RStudio
\end{itemize}
\end{frame}

%----------------------------------------------------------

\begin{frame}[fragile]{R sistema}
R susideda iš dviejų komponentų:
\begin{enumerate}
\item Bazinė R sistema su standartiniais paketais: stats, graphics, grDevices, utils, datasets, methods, base  (galima pamatyti įvedus komandą search() )
\item Visų kitų paketų
\end{enumerate}
\begin{itemize}
\item dauguma R paketų saugomi CRAN (Comprehensive R Archive Network), iš kur atsisiunčiamas ir pats R
\item available.packages() funkcija, kuri surenką visą informaciją apie ezistuojančius R paketus @CRAN
\begin{lstlisting}
a <- available.packages()
length(a)
\end{lstlisting}
\item Šiuo metu : 230180 paketai
\item Taip pat galima instaliuoti ir paketus esančius @GitHub
\end{itemize}
\end{frame}

%----------------------------------------------------------

\begin{frame}{Kur rasti pagalbą}
99.99\%, na ką jau, 150\% visi susidursite su problemomis, kai kažkas neveiks kaip norite, kai R praneš apie klaidas ir t.t. ir kai patys nežinosite ką daryti toliau. Todėl toks "pagalbos eiliškumas": 
\begin{enumerate}
\item R, R paktetų, Git, GitHub dokumentacija, "help" funkcija
\item Google (Ctrl+C Ctrl+V error code) 99.99\% kažkas jau tą problemą turėjo
\item Kursiokai / Mokslo grupė
\item \href{https://stackoverflow.com/}{\color{blue}{https://stackoverflow.com/}} (žr. sekanti skaidrė)
\item Dėstytojas (žr. sekanti skaidrė)
\end{enumerate}
\end{frame}


%----------------------------------------------------------

\begin{frame}{Stackoverflow}
\begin{itemize}
\item \href{https://stackoverflow.com/}{\color{blue}{https://stackoverflow.com/}}
\item Kokius konkrečiai žingsnius atlikote
\item Kokio rezultato tikitės
\item Kokį rezultatą gaunate
\item Kokią R versiją, kokius paketus naudojate (retai: kokia operacinė sistema)
\item Visada geriausia aprašyti problemą, bei pateikti visą kodą, leidžiantį atkartoti Jūsų problemą
\item Antraštė turėtų būti trumpa ir aiški
\end{itemize}
\end{frame}

%----------------------------------------------------------
\subsection{R Input Output}
%----------------------------------------------------------

%----------------------------------------------------------

\begin{frame}[fragile]{R Input Output}
"<-" yra priskyrimo operatorius, ">" promt (CLI buvo \$)
\begin{lstlisting}
> x <- 1
> print(x)
[1] 1
> msg <- "hello world"
> print(msg)
[1] "hello world"
\end{lstlisting}

Komentarai atskiriami su \# viskas į dešinę nuo \# ignoruojama (toje eilutėje)
\begin{lstlisting}
> msg <- "hello world" #pirma žinute
> msg # autoprint prints values without entering command print()
[1] "hello world"
> x <- 
+ 
\end{lstlisting}
ESC arba pabaigti įvesti. [1] indikuoja vektoriaus reikšmės numerį
\end{frame}

%----------------------------------------------------------

\begin{frame}[fragile]{R Input Output}
Operatorius : sukuria eiles (sequence)
\begin{lstlisting}
> x <- 1:30
> x
 [1]  1  2  3  4  5  6  7  8  9 10 11 12 13 14 15 16 17 18
[19] 19 20 21 22 23 24 25 26 27 28 29 30
\end{lstlisting}
\end{frame}
%----------------------------------------------------------
\subsection{Objektų tipai}
%----------------------------------------------------------

\begin{frame}[fragile]{R objektai}
R turi 5 bazinius objektų tipus / klases (\textit{atomic classes}):
\begin{itemize}
\item charackter: "vilnius", "amžius"
\item numeric: 1, 4.5, -1.1...
\item integer 1,2,3,4,5 (sveikas skaičius)
\item complex 1+2i
\item logical TRUE /FALSE arba T/F
\end{itemize}
Dažniausiai naudojamas vektorius, kuriame gali būti tik tos pačios klasės objektai. Tuščią vektorių galima sukurti su komanda vector()\\
list (sąrašas) gali talpinti įvairių klasių objektus.
\end{frame}

%----------------------------------------------------------

\begin{frame}[fragile]{R objektai}
\begin{itemize}
\item Skaičius R supranta kaip numeric klasės objektus
\item Jeigu reikia pilno skaičiaus (integer) tada skaičių reikia pabaigti su L raide 
\item Inf suprantamas kaip begalybė
\item NAN ("not a number"), arba  trūkstama reikšmė
\begin{lstlisting}
> x<-2L
> x
[1] 2
> x <-2.1L
Warning message:
integer literal 2.1L contains decimal; using numeric value 
> Inf
[1] Inf
> 1/Inf
[1] 0
> 0/0
[1] NaN
\end{lstlisting}
\end{itemize}
\end{frame}

%----------------------------------------------------------

\begin{frame}[fragile]{Atributai}
R objektai gali turėti atributus.
\begin{itemize}
\item names, dimnames
\item dimensions (e.g matricos)
\item class (numeric, charackter)
\item length
\item kiti vartotojo priskirti atributai
\item attributes() leidžia nustatyti / keisti objekto atributus
\end{itemize}
\end{frame}

%----------------------------------------------------------

\begin{frame}[fragile]{Vektorių sukūrimas}

\begin{itemize}
\item c() funkcija leidžia sukurti objektų vektorius 
\item c() iš concatenate 
\begin{lstlisting}
> x <- c(0.2 , 0.6) # numeric class
> x <- c(TRUE, FALSE) #logical class
> x <- c(T, F) #logical class
> x <- c("a", "b", "c") #character class
> x <- 1:5  #integer class
> x <- c(1+0i, 2+4i) #complex class
\end{lstlisting}
\item galima sukurti tuščią vektorių, nurodant kokios klasės objektai jame bus ir kokia vektoriaus dimensija
\begin{lstlisting}
> x <- vector(mode="numeric", length = 8)
> x
[1] 0 0 0 0 0 0 0 0
\end{lstlisting}
\end{itemize}
\end{frame}

%----------------------------------------------------------

\begin{frame}[fragile]{Vektorių sujungimas (\textit{coersion})}
\begin{itemize}
\item jeigu su c() sujungiami skirtingų klasių objektai, R priskiria bendriausią klasę visiems vektoriuje esantiems objektams
\begin{lstlisting}
> x <- c(0.2 , "a") # character class
> x <- c(TRUE, FALSE, 3) #numeric class
> x <- c("a", TRUE, FALSE) #character class
\end{lstlisting}
\item TRUE=1, FALSE=0
\item procesas kuris vyksta vadinamas \textit{coersion}
\end{itemize}
\end{frame}

%----------------------------------------------------------

\begin{frame}[fragile]{Vektorių sukūrimas}
\begin{itemize}
\item Vektorius galima rankiniu būdu priskirti tam tikrai klasei 
\begin{lstlisting}
> x<-10:19
> class(x)
[1] "integer"
> as.numeric(x)
 [1]  0  1  2  3  4  5  6  7  8  9 10
> as.logical(x)
 [1] FALSE  TRUE  TRUE  TRUE  TRUE  TRUE  TRUE  TRUE  TRUE
[10]  TRUE  TRUE
> as.character(x)
 [1] "0"  "1"  "2"  "3"  "4"  "5"  "6"  "7"  "8"  "9" 
[11] "10"
> as.factor(x)
 [1] 0  1  2  3  4  5  6  7  8  9  10
Levels: 0 1 2 3 4 5 6 7 8 9 10
\end{lstlisting}
\end{itemize}
\end{frame}

%----------------------------------------------------------

\begin{frame}[fragile]{Vektorių sukūrimas}
\begin{itemize}
\item tačiau nelogiški manualūs priskyrimai generuos NAs
\begin{lstlisting}
> x<- c("a", "b", "c")
> as.numeric(x)
[1] NA NA NA
Warning message:
NAs introduced by coercion 

> as.logical(x)
[1] NA NA NA

> as.complex(x)
[1] NA NA NA
Warning message:
NAs introduced by coercion 
\end{lstlisting}
\end{itemize}
\end{frame}

%----------------------------------------------------------

\begin{frame}[fragile]{List - sąrašas}
\begin{itemize}
\item List gali talpinti įvairių klasių objektus
\begin{lstlisting}
> x <- list(1.2, 3L, TRUE, F, 1+4i, 1:3) 
> x
[[1]]
[1] 1.2
[[2]]
[1] 3
[[3]]
[1] TRUE
[[4]]
[1] FALSE
[[5]]
[1] 1+4i
[[6]]
[1] 1 2 3
\end{lstlisting}
\item $[[nr]]$  nurodo list objekto numerį
\end{itemize}
\end{frame}

%----------------------------------------------------------

\begin{frame}[fragile]{Matricos}
\begin{itemize}
\item Matricos, tai tas pats vektoriaus objektas, tačiau turintis dimensijos nustatymus
\begin{lstlisting}
> x <- matrix(nrow = 3, ncol = 3)
> x
     [,1] [,2] [,3]
[1,]   NA   NA   NA
[2,]   NA   NA   NA
[3,]   NA   NA   NA

> dim(x)
[1] 3 3

> attributes(x)
$dim
[1] 3 3
\end{lstlisting}
\end{itemize}
\end{frame}

%----------------------------------------------------------

\begin{frame}[fragile]{Matricos}
\begin{itemize}
\item Matricos užpildomos stulpeliniu būdų, jeigu nenurodoma kitaip
\item ?matrix parodo funkcijos manual
\begin{lstlisting}
> m <- matrix(1:9, nrow = 3, ncol = 3)
> m
     [,1] [,2] [,3]
[1,]    1    4    7
[2,]    2    5    8
[3,]    3    6    9

>? matrix

> m <- matrix(1:9, nrow = 3, ncol = 3, byrow = TRUE)
> m
     [,1] [,2] [,3]
[1,]    1    2    3
[2,]    4    5    6
[3,]    7    8    9

\end{lstlisting}
\end{itemize}
\end{frame}

%----------------------------------------------------------

\begin{frame}[fragile]{Matricos}
\begin{itemize}
\item Vektorius be dimensijų yra paprastas vektorius
\item Tačiau vektoriui galima suteikti dimensijas post factum, tada vektorius tampa matrica
\begin{lstlisting}
> v <- 1:12
> v
 [1]  1  2  3  4  5  6  7  8  9 10 11 12
> dim(v) <-c(4,3)
> v
     [,1] [,2] [,3]
[1,]    1    5    9
[2,]    2    6   10
[3,]    3    7   11
[4,]    4    8   12
\end{lstlisting}
\end{itemize}
\end{frame}

%----------------------------------------------------------

\begin{frame}[fragile]{cbind , rbind}
\begin{itemize}
\item cbind (columnbind) ir rbind (rowbind) iš atskirų vektorių sukuria matricas
\begin{lstlisting}
> x <-1:3
> y <- 20:22
> cbind(x,y)
     x  y
[1,] 1 20
[2,] 2 21
[3,] 3 22
> rbind(x,y)
  [,1] [,2] [,3]
x    1    2    3
y   20   21   22
\end{lstlisting}
\end{itemize}
\end{frame}

%----------------------------------------------------------

\begin{frame}[fragile]{cbind , rbind}
\begin{itemize}
\item Tačiau jeigu vektorių dydis ne toks pats... r coersion'a 
\begin{lstlisting}
> x <-1:3
> y <- 1:5
> cbind(x,y)
     x y
[1,] 1 1
[2,] 2 2
[3,] 3 3
[4,] 1 4
[5,] 2 5
Warning message:
In cbind(x, y) : number of rows of result is not a multiple of vector length (arg 1)
> rbind(x,y)
  [,1] [,2] [,3] [,4] [,5]
x    1    2    3    1    2
y    1    2    3    4    5
Warning message:
In rbind(x, y) :number of columns of result is not a multiple of vector length (arg 1)
\end{lstlisting}
\end{itemize}
\end{frame}


%----------------------------------------------------------

\begin{frame}[fragile]{Faktoriai}
\begin{itemize}
\item Faktorių klasė skirta kategoriniams kintamiesiems (vardiniai, ranginiai)
\item Faktoriai yra svarbūs modeliuojant bei kartais grafikams

\begin{lstlisting}
> x <-factor(c("taip", "ne", "taip", "taip", "ne"))
> x
[1] taip ne   taip taip ne  
Levels: ne taip
> table(x)
x
  ne taip 
   2    3 
> unclass(x)
[1] 2 1 2 2 1
attr(,"levels")
[1] "ne"   "taip"
\end{lstlisting}
\end{itemize}
\end{frame}

%----------------------------------------------------------

\begin{frame}[fragile]{Faktoriai}
\begin{itemize}
\item Lygiai pagal alfabetinį eiliškumą pasirodantį vektoriuje, arba nurodoma manualiai

\begin{lstlisting}
> x <-factor(c("girtas", "blaivus", "girtas" , "girtas" , "blaivus"),levels=c("girtas", "blaivus"))
> x
[1] girtas  blaivus girtas  girtas  blaivus
Levels: girtas blaivus
> table(x)
x
 girtas blaivus 
      3       2  
\end{lstlisting}
\end{itemize}
\end{frame}

%----------------------------------------------------------

\begin{frame}[fragile]{Trūkstami skaičiai}
\begin{itemize}
\item Trūkstami skaičiai pateikiami kaip NA
\item Neapibrėžtos matematinės reikšmės NaN
\item is.na() testuoja ar egzistuoja NA
\item is.nan() testuoja ar egzistuoja NaN
\item NAN gali turėti klases (integer, numeric)
\item NaN yra NA, bet NA nėra NaN
\end{itemize}
\end{frame}

%----------------------------------------------------------

\begin{frame}[fragile]{Trūkstami skaičiai}
\begin{itemize}
\item is.na() ir is.nan() komandos pateikia vektorių, kuriame atspindimas testavimo rezultatas
\begin{lstlisting}
> x <- c(1,2,NA,4,5,6)
> is.na(x)
[1] FALSE FALSE  TRUE FALSE FALSE FALSE
> is.nan(x)
[1] FALSE FALSE FALSE FALSE FALSE FALSE

> x <- c(1,2,NA,4,NaN,6)
> is.na(x)
[1] FALSE FALSE  TRUE FALSE  TRUE FALSE
> is.nan(x)
[1] FALSE FALSE FALSE FALSE  TRUE FALSE
\end{lstlisting}
\end{itemize}
\end{frame}

%----------------------------------------------------------

\begin{frame}[fragile]{Data frames}
\begin{itemize}
\item Data frames naudojami laikyti tabelinius duomenis
\item Iš esmės tai specialus atvejis List, kuriame kiekvienas stulpelis turi būti to paties ilgio
\item Kiekvienas stulpelis gali talpinti vis kitos klasės duomenis
\item Specialūs data frames atributai
\begin{itemize}
\item rownames
\item colnames
\end{itemize}
\item Dažnai sukuriamos nuskaitant duomenis pvz., read.table() arba read.csv()
\item Galima pakeisti į matricą su as.matrix()
\item Tuščią \textit{data frame} galima sukurti su data.frame()
\end{itemize}
\end{frame}

%----------------------------------------------------------

\begin{frame}[fragile]{Data frames}
\begin{itemize}
\item Data frames naudojami laikyti tabelinius duomenis
\begin{lstlisting}
> x <-data.frame(FName=c("Ana", "Maria", "John", "Peter"), Grades=c(9,10,7,8))
> x
  FName Grades
1   Ana      9
2 Maria     10
3  John      7
4 Peter      8
> nrow(x)
[1] 4
> ncol(x)
[1] 2
> rownames(x)
[1] "1" "2" "3" "4"
> colnames(x)
[1] "FName"  "Grades"
\end{lstlisting}
\end{itemize}
\end{frame}
%----------------------------------------------------------

\begin{frame}[fragile]{Data frames}
\begin{itemize}
\item Data frames galima priskirti eilučių ir stulpelių pavadinimus
\begin{lstlisting}
> y <- data.frame(1:3)
> y
  X1.3
1    1
2    2
3    3
> colnames(y) <- "NR"
> y
  NR
1  1
2  2
3  3
> rownames(y) <- c("alpha", "beta", "gama")
> y
      NR
alpha  1
beta   2
gama   3
\end{lstlisting}
\end{itemize}
\end{frame}

%----------------------------------------------------------

\begin{frame}[fragile]{List names}
\begin{itemize}
\item List irgi gali turėti pavadinimus
\begin{lstlisting}
> x <- list (a=1, b=2, c=c(1:3))
> x
$a
[1] 1

$b
[1] 2

$c
[1] 1 2 3
\end{lstlisting}
\end{itemize}
\end{frame}

%----------------------------------------------------------

\begin{frame}[fragile]{Matrix names}
\begin{itemize}
\item Matricos irgi gali turėti pavadinimus, tik čia tai dimnames()
\begin{lstlisting}
> m <-matrix(1:4, nrow = 2, ncol=2)
> m
     [,1] [,2]
[1,]    1    3
[2,]    2    4
> dimnames(m) <- list(c("a", "b"), c("c", "d"))
> m
  c d
a 1 3
b 2 4
\end{lstlisting}
\end{itemize}
\end{frame}

%----------------------------------------------------------

\begin{frame}[fragile]{Vektorių vardai}
\begin{itemize}
\item Vektorių įverčiams irgi galima priskirti pavadinimus
\begin{lstlisting}
> x <- 1:3
> x
[1] 1 2 3
> names(x) <- c("a", "b", "c")
> x
a b c 
1 2 3
> str(x)
 Named int [1:3] 1 2 3
 - attr(*, "names")= chr [1:3] "a" "b" "c"
\end{lstlisting}
\end{itemize}
\end{frame}

%----------------------------------------------------------
\subsection{Duomenų importas į R}
%----------------------------------------------------------

\begin{frame}[fragile]{Duomenų importas į R}
Pagrindinės funkcijos, kurios apdeda importuoti duomenis į R
\begin{itemize}
\item read.table(), read.csv() tabelinių duomenų importavimui
\item readLines nuskaityti tekstą (pvz. .txt, .html)
\item source() importuoti R kodo failus
\item dget() importuoti R kodo failus
\item load() importavimas išsaugotų darbolaukių (workspace)
\item unserialize(), importavimas R objektų binarinėje formoje
\end{itemize}
\end{frame}

%----------------------------------------------------------

\begin{frame}[fragile]{Duomenų eksportas iš R}
Pagrindinės funkcijos, kurios apdeda eksportuoti duomenis į R
\begin{itemize}
\item write.table(), write.csv()
\item writeLines()
\item dump()
\item dput()
\item save()
\item serialize()
\end{itemize}
\end{frame}

%----------------------------------------------------------

\begin{frame}[fragile]{read.table()}
read.table() pagrindiniai argumentai
\begin{itemize}
\item file, nuskaitomo failo pavadinimas
\item header, loginis indikatorius, ar egzistuoja stulpelių pavadinimai
\item sep, nudoro kaip atskirti stulpeliai
\item colClasses, vektorius, nurodantis skirtingas stulpelių klases
\item nrows, eilučių skaičius duomenyse
\item comment.char, nurodo kaip žymimi komentarai faile
\item skip, skaičius, kiek eilučių nuo viršaus praleisti
\item stringsAsFactors, ar character variables turėtų būti pakeisti į faktorius
\item pilnas sąrašas ?read.table
\end{itemize}
\end{frame}

%----------------------------------------------------------

\begin{frame}[fragile]{PVZ: read.table() + CLI + ping}

\begin{itemize}
\item R'e pasitikriname kur esame: getwd()
\item Jeigu reikia, su setwd() pasikeičiame į folderį Sxxx/R/data
\item CLI užrašome komandą: 
\begin{lstlisting}
ping www.lithuanian-economy.net > c/Users/studentas/Dekstop/Sxxxx/R/data/LE-ping.txt
\end{lstlisting}
\item leidžiame paveikti surinkti duomenų ir nutraukiam su Ctrl+C
\item su Sublime pasižiūrime kaip atrodo duomenys (pas kiekvieną kiek kitaip)
\item 1 eilutės nereikia, kaip ir paskutinių...
\item pvz: 
\begin{lstlisting}
data <-read.table("LE.txt", skip=1, header = FALSE, sep=" ", nrows = 46, comment.char="")
\end{lstlisting}
\end{itemize}
\end{frame}

%----------------------------------------------------------
\begin{frame}[fragile]{PVZ: read.table() + CLI + ping}
\begin{itemize}
\item Išdaliname 8 stulpelį į dvi dalis
\item Išdalinimas tampa list
\item Iš list paverčiame dataframe  (apie do.call pakalbėsime vėliau)
\item Paverčiame df antrą stulpelį į numeric
\item su hist() nupiešiame paprastą histogramą
\begin{lstlisting}
time <-strsplit(as.character(data$V8),'=',fixed=TRUE)
df <-  data.frame(do.call(rbind, time))
df$X2 <- as.numeric(df$X2)
str(df)
hist(df$X2)
\end{lstlisting}
\item PVZ baigtas
\end{itemize}
\end{frame}


%----------------------------------------------------------
\begin{frame}[fragile]{Didelių failų nuskaitymas}
\begin{itemize}
\item Kiek reikia RAM norimiems duomenims:
\begin{itemize}
\item 1 numeric įrašas = 8 bytes
\item 1 000 000 eilučių ir 20 stulpelių 
\item $1000000 \times 200 \times 8=1880000000$
\item 1024 bytes = KB, 1024 KB = 1 MB, $\frac{bytes}{2^{20}}=MB$, $\frac{bytes}{2^{30}}=GB$
\item $\frac{1880000000}{2^{20}}=1792.9Mb$ arba 1.75 GB
\item nykščio taisyklė, reikia dvigubai daugiau RAM!
\end{itemize}
\item Naudoti comment.char=""
\item Galima padėti R geriau optimizuoti RAM nurodant nrows=...
\begin{lstlisting}
initial <- read.table("data.txt", nrows=100)
classes <- sapply(initial, class)
df <- read.table("data.txt", colClasses=classes,
							comment.char="")
\end{lstlisting}
\end{itemize}
\end{frame}

%----------------------------------------------------------
\begin{frame}[fragile]{Tekstiniai formatai}
\begin{itemize}
\item dump(), dput() išsaugo duomenis tekstiniu formatu, bet su meta duomenimis
\item dput() skirtas vienam failui
\item dump() skirtas vienam arba daugiau failų
\item Tekstiniai formatai idealus naudojant VCS
\item Tekstiniai formatai yra universalūs, todėl iš esmes atsparūs "zeitgeist"
\item Minusas, jog tekstiniai formatai užima daugiau vietos
\end{itemize}
\end{frame}

%----------------------------------------------------------
\begin{frame}[fragile]{dump() ir source()}
\begin{lstlisting}
AirQualityUCI <- read.csv("AirQualityUCI.csv", 
                          sep=";", 
                          header = TRUE, 
                          stringsAsFactors = FALSE, 
                          comment.char = "")
household_power_consumption <- read.table("household_power_consumption.txt", 
                                          sep=";", 
                                          header = TRUE, 
                                          stringsAsFactors = FALSE, 
                                          comment.char = "")

list_files <-ls()
dump(list_files, file="duomenys.R")
dump(c("AirQualityUCI", 
       "household_power_consumption"), 
     file="duomenys.R")
rm(list=ls())
source("duomenys.R")
\end{lstlisting}
\end{frame}

%----------------------------------------------------------
\begin{frame}[fragile]{dump() ir source()}
\begin{itemize}
\item dump veikia gerai, kol failai nėra labai dideli (<100mb)
\item Reikia patiems įsivertinti, kas yra greičiau dump() + source()
\item ar visgi read.table() + visos komandos...
\end{itemize}
\end{frame}

%----------------------------------------------------------
\subsection{R ir išorinis pasaulis}
%----------------------------------------------------------

\begin{frame}[fragile]{R ir išorinis pasaulis}
Kai nuskaitome failą, R naudojasi file funkcija, kad sudarytų ryšį su norimu failu
\begin{itemize}
\item file, atidaro ryšį su failu
\begin{lstlisting}
function (description = "", open = "", blocking = TRUE, encoding = getOption("encoding"), 
    raw = FALSE, method = getOption("url.method", "default")) 
\end{lstlisting}

\item description - failo pavadinimas
\item open - "r" (read only), "w" (writing), "a" (appending), "rb", "wb", "ab" (binarinėje formoje)
\end{itemize}
\end{frame}

%----------------------------------------------------------
\begin{frame}[fragile]{R ir išorinis pasaulis}
\begin{itemize}
\item abu variantai tolygūs, nes funkcija read.table viduje naudojasi file funkcija
\end{itemize}
\begin{lstlisting}
con <- file("LE.txt", "r")
data <- read.table(con, sep=" ", 
                   skip = 1, 
                   nrows = 156, 
                   stringsAsFactors = FALSE, 
                   comment.char = "", 
                   header = FALSE)
close(con)

data <- read.table("LE.txt", 
                   sep=",", 
                   skip = 1, 
                   nrows = 60, 
                   stringsAsFactors = FALSE, 
                   comment.char = "")
\end{lstlisting}

\end{frame}





%----------------------------------------------------------
\begin{frame}[fragile]{R ir išorinis pasaulis}
Galima atidaryti ryšį ir su zipintais failais
\begin{itemize}
\item gzfile, atidro ryšį su .gzip failu
\item bzfile, atidro ryšį su .bzip2 failu
\begin{lstlisting}
con <- gzfile("census-income.data.gz")
data <- read.table(con, sep=",", 
                   nrows = 100, 
                   stringsAsFactors = FALSE, 
                   comment.char = "", 
                   header = FALSE)

close(con)
\end{lstlisting}
\end{itemize}
\end{frame}


%----------------------------------------------------------
\begin{frame}[fragile]{R ir išorinis pasaulis}
\begin{itemize}
\item url, atidro ryšį su web tinklapiu
\item galima nuskaityti pasirinkto tinklapio html kodą
\begin{lstlisting}
con <- url("http://www.delfi.lt", "r")
data <- readLines(con)
close(con)
\end{lstlisting}
\end{itemize}
\end{frame}


%----------------------------------------------------------
\subsection{Subsetting}
%----------------------------------------------------------
\begin{frame}[fragile]{Subsetting}
Pagrindiniai operatoriai leidžiantys pasirinkti dalį R objektų
\begin{itemize}
\item $[...]$ visada duoda objektą tos pačios klasės, galima pasirinkti daugiau nei vieną elementą
\item $[[...]]$ vieno elemento iš list arba dataframe pasirinkimui
\item \$ leidžia pasirinkti pagal pavadinimus (pagal col.names)
\end{itemize}
\end{frame}

%----------------------------------------------------------
\begin{frame}[fragile]{Subsetting}
\begin{itemize}
\item skaitinis indeksas
\item loginis indeksas
\end{itemize}
\begin{lstlisting}

x <- c("a", "b", "c", "d")

#skaitinis
x[1]
x[2]
x[1:3]

#loginis
x[x>"b"]
rule <- x>"b"
rule
x[rule]
\end{lstlisting}
\end{frame}

%----------------------------------------------------------
\begin{frame}[fragile]{Subsetting}
\begin{itemize}
\item Subsetting naudojant list objektą
\item Pradžiai pasidarome šį objektą:
\end{itemize}
\begin{lstlisting}
> x <- list(grades=1:10, 
			names=c("Ana", "Maria", "John", "Peter"),
			course=c("1gr", "2gr"))

\end{lstlisting}
\end{frame}

%----------------------------------------------------------
\begin{frame}[fragile]{Subsetting list}
\begin{lstlisting}
> x[1]
$grades
 [1]  1  2  3  4  5  6  7  8  9 10

> x[[1]]
 [1]  1  2  3  4  5  6  7  8  9 10

> x$names
[1] "Ana"   "Maria" "John"  "Peter"

> x["names"]
$names
[1] "Ana"   "Maria" "John"  "Peter"

> x[["names"]]
[1] "Ana"   "Maria" "John"  "Peter"
\end{lstlisting}
\end{frame}

%----------------------------------------------------------
\begin{frame}[fragile]{Subsetting list}
\begin{lstlisting}
> x[c(1,3)]
$grades
 [1]  1  2  3  4  5  6  7  8  9 10

$course
[1] "1gr" "2gr"

> x[[c(1,3)]]
[1] 3

\end{lstlisting}
\end{frame}

%----------------------------------------------------------
\begin{frame}[fragile]{Subsetting list}
\begin{lstlisting}
> kint <- "names"

> x[kint]
$names
[1] "Ana"   "Maria" "John"  "Peter"

> x[[kint]]
[1] "Ana"   "Maria" "John"  "Peter"

> x$kint
NULL
\end{lstlisting}
\end{frame}


%----------------------------------------------------------
\begin{frame}[fragile]{Subsetting list}
\begin{lstlisting}
> x <- list(grades=1:10, names=c("Ana", "Maria", "John", "Peter"), 
+           course=c("1gr", "2gr"))

> x[[2]]
[1] "Ana"   "Maria" "John"  "Peter"

> x[[c(2,2)]]
[1] "Maria"

> x[[2]][[2]]
[1] "Maria"
\end{lstlisting}
\end{frame}

%----------------------------------------------------------
\begin{frame}[fragile]{Subsetting}
\begin{itemize}
\item Subsetting naudojant matricą (i,j) 
\item Subsetting su $[]$ duoda vektorių, ne matricą!
\end{itemize}
\begin{lstlisting}
> m <- matrix(1:9, nrow=3, ncol=3)
> m
     [,1] [,2] [,3]
[1,]    1    4    7
[2,]    2    5    8
[3,]    3    6    9

> m[1,1]

[1] 1
> m[3,3]

[1] 9
> m[2,]
[1] 2 5 8

> m[,3]
[1] 7 8 9
\end{lstlisting}
\end{frame}

%----------------------------------------------------------
\begin{frame}[fragile]{Subsetting}
\begin{itemize}
\item Subsetting naudojant matricą (i,j) 
\item Subsetting su $[]$ duoda vektorių, ne matricą, todėl drop=FALSE
\end{itemize}
\begin{lstlisting}
> m <- matrix(1:9, nrow=3, ncol=3)

> m[1,1, drop=FALSE]
     [,1]
[1,]    1

> m[2,, drop=FALSE]
     [,1] [,2] [,3]
[1,]    2    5    8

> m[,3, drop=FALSE]
     [,1]
[1,]    7
[2,]    8
[3,]    9
\end{lstlisting}
\end{frame}


%----------------------------------------------------------

\begin{frame}[fragile]{NA išvalymas}
\begin{itemize}
\item Kartais duomenyse yra NA
\end{itemize}
\begin{lstlisting}
> x <- c(1,2,3,NA,5,6,NA,8)
> y <-c("a", "b", "c", NA, NA, "f", "g" , "h")

> is.na(x)
[1] FALSE FALSE FALSE  TRUE FALSE FALSE  TRUE FALSE

> trukst_vek <- is.na(x)

> x[trukst_vek]
[1] NA NA

> x[!trukst_vek]
[1] 1 2 3 5 6 8
\end{lstlisting}
\end{frame}

%----------------------------------------------------------

\begin{frame}[fragile]{NA išvalymas}
\begin{itemize}
\item complete.cases() duoda loginį vektorių, su pozicijomis, kuriose nėra NA
\end{itemize}
\begin{lstlisting}
> complete.cases(x,y)
[1]  TRUE  TRUE  TRUE FALSE FALSE  TRUE FALSE  TRUE

> x[complete.cases(x,y)]
[1] 1 2 3 6 8

> y[complete.cases(x,y)]
[1] "a" "b" "c" "f" "h"
\end{lstlisting}
\end{frame}

%----------------------------------------------------------

\begin{frame}[fragile]{NA išvalymas}
\begin{itemize}
\item Su complete.cases() galima išvalyti ir dataframe
\end{itemize}
\begin{lstlisting}
> library(datasets)
> airquality[1:6,]
  Ozone Solar.R Wind Temp Month Day
1    41     190  7.4   67     5   1
2    36     118  8.0   72     5   2
3    12     149 12.6   74     5   3
4    18     313 11.5   62     5   4
5    NA      NA 14.3   56     5   5
6    28      NA 14.9   66     5   6

> airquality[complete.cases(airquality),][1:6,]
  Ozone Solar.R Wind Temp Month Day
1    41     190  7.4   67     5   1
2    36     118  8.0   72     5   2
3    12     149 12.6   74     5   3
4    18     313 11.5   62     5   4
7    23     299  8.6   65     5   7
8    19      99 13.8   59     5   8
\end{lstlisting}
\end{frame}

%----------------------------------------------------------

\begin{frame}[fragile]{NA išvalymas}
\begin{itemize}
\item kita alternatyva: na.omit()
\end{itemize}
\begin{lstlisting}
> library(datasets)
> airquality[1:6,]
  Ozone Solar.R Wind Temp Month Day
1    41     190  7.4   67     5   1
2    36     118  8.0   72     5   2
3    12     149 12.6   74     5   3
4    18     313 11.5   62     5   4
5    NA      NA 14.3   56     5   5
6    28      NA 14.9   66     5   6

> na.omit(airquality)[1:6,]
  Ozone Solar.R Wind Temp Month Day
1    41     190  7.4   67     5   1
2    36     118  8.0   72     5   2
3    12     149 12.6   74     5   3
4    18     313 11.5   62     5   4
7    23     299  8.6   65     5   7
8    19      99 13.8   59     5   8
\end{lstlisting}
\end{frame}


%----------------------------------------------------------

\begin{frame}[fragile]{NA išvalymas}
O bet tačiau...
\begin{itemize}
\item O bet tačiau... ypatingai atliekant apklausas, visada nutiks taip, jog dalis respondentų neatsakys į kuriuos nors pavienius klausimus (pvz., nesupras klausimo)
\item Visos obzervacijos panaikinimas gali būti labai "brangus", ypač turint mažą n
\item Todėl geriau atliekant skaičiavimus su R, funkcijoms nurodyti kaip apeiti NA
\item Alternatyva, pakeisti NA pvz 0, arba vidutine kintamojo reikšme. Tačiau tai būtina protokoluoti ir nurodyti tyrime / tyrimo meta apraše

\end{itemize}
\end{frame}

%----------------------------------------------------------

\begin{frame}[fragile]{Vektorizuotos operacijos}
\begin{itemize}
\item R skaičiavimus atlieka vektorizuojant savo objektus
\end{itemize}
\begin{lstlisting}
> x<- 1:5; y<-3:7; z<- 1:2

> x+y
[1]  4  6  8 10 12
> x*y
[1]  3  8 15 24 35
> x/y
[1] 0.3333333 0.5000000 0.6000000 0.6666667 0.7142857

> x+z
[1] 2 4 4 6 6
Warning message:
In x + z : longer object length is not a multiple of shorter object length

> x*z
[1] 1 4 3 8 5
Warning message:
In x * z : longer object length is not a multiple of shorter object length
\end{lstlisting}
\end{frame}

%----------------------------------------------------------

\begin{frame}[fragile]{Matricos}
\begin{lstlisting}
> x<-matrix(1:4,2,2); y<-matrix(rep(10,4),2,2); s<-matrix(1:2,nrow=2)

> x
     [,1] [,2]
[1,]    1    3
[2,]    2    4
> y
     [,1] [,2]
[1,]   10   10
[2,]   10   10
> x*y
     [,1] [,2]
[1,]   10   30
[2,]   20   40
> x+y
     [,1] [,2]
[1,]   11   13
[2,]   12   14
\end{lstlisting}
\end{frame}

\begin{frame}[fragile]{Matricos}
\begin{lstlisting}
> x<-matrix(1:4,2,2); y<-matrix(rep(10,4),2,2); s<-matrix(1:2,nrow=2)

> x%*%y
     [,1] [,2]
[1,]   40   40
[2,]   60   60
> x%*%s
     [,1]
[1,]    7
[2,]   10
\end{lstlisting}
\end{frame}
%----------------------------------------------------------
\end{document}