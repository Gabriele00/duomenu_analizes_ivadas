\documentclass[11pt,xcolor=table]{beamer}
\usetheme{CambridgeUS}
\usecolortheme{dolphin}

\usepackage[utf8]{inputenc}
\usepackage[L7x]{fontenc}
\usepackage[lithuanian]{babel}
\usepackage{lmodern} %be neveikia bold, italic etc


\usepackage{amsmath}
\usepackage{amsfonts}
\usepackage{amssymb}

\usepackage{graphicx}
\graphicspath{{./figures/}}

\usepackage{adjustbox}
\usepackage{bm}
\usepackage{subcaption}

\usepackage{multirow} % for multirow tables
\setbeamertemplate{caption}[numbered] % for numbering captions

\usepackage{forest}


\usepackage{listings}
\usepackage{color}
\usepackage{textcomp}
\definecolor{listinggray}{gray}{0.9}
\definecolor{lbcolor}{rgb}{0.9,0.9,0.9}
\lstset{
	backgroundcolor=\color{lbcolor},
	tabsize=4,
	rulecolor=,
	language=matlab,
        basicstyle=\scriptsize,
        upquote=true,
        aboveskip={0.5\baselineskip},
        belowskip={0.5\baselineskip},
        columns=fixed,
        showstringspaces=false,
        extendedchars=false,
        breaklines=true,
        prebreak = \raisebox{0ex}[0ex][0ex]{\ensuremath{\hookleftarrow}},
        frame=single,
        showtabs=false,
        showspaces=false,
        showstringspaces=false,
        identifierstyle=\ttfamily,
        keywordstyle=\color[rgb]{0,0,1},
        commentstyle=\color[rgb]{0.133,0.545,0.133},
        stringstyle=\color[rgb]{0.627,0.126,0.941}
}


\usepackage{tikz, pgfplots}
\usetikzlibrary{patterns,decorations.pathreplacing}


\author{Justas Mundeikis}
\title{Duomenų analizės įvadas}
\subtitle{1. Dalis}
%\setbeamercovered{transparent} 
%\setbeamertemplate{navigation symbols}{} 
%\logo{} 
%\institute{Vilniaus Universitetas, EVAF} 
%\date{} 
%\subject{} 


\setbeamerfont{subsection in toc}{size=\scriptsize}
%----------------------------------------------------------
\begin{document}
%----------------------------------------------------------

\begin{frame}
\titlepage
\end{frame}

%----------------------------------------------------------

\begin{frame}{1. Dalies turinys}
\tableofcontents
\end{frame}

%----------------------------------------------------------
\section{R}
%----------------------------------------------------------

%----------------------------------------------------------
\subsection{Įvadas i R}
%---------------------------------------------------------

\begin{frame}{Įvadas į R}
\begin{itemize}
\item R istorija
\item R instaliavimas
\end{itemize}
\end{frame}

%----------------------------------------------------------

\begin{frame}{R ir RStudio instaliavimas}
\begin{itemize}
\item R reikia instaliuoti iš CRAN
\item \href{https://cran.r-project.org/}{\textcolor{blue}{https://cran.r-project.org/}}
\item Paleidžiame R 
\item Tam kad būtų lengviau dirbti su R, turėti aibę papildomų funkcijų, instaliuojame RStudio
\item \href{https://www.rstudio.com/products/rstudio/download/}{\textcolor{blue}{https://www.rstudio.com/products/rstudio/download/}}
\item Startuojame RStudio
\end{itemize}
\end{frame}

%----------------------------------------------------------

\begin{frame}[fragile]{R paketai}
\begin{itemize}
\item dauguma R paketų saugomi CRAN (Comprehensive R Archive Network), iš kur atsisiunčiamas ir pats R
\item basinė R versija turi tik keletą naudingų paketų
\item available.packages() funkcija, kuri surenką visą informaciją apie ezistuojančius R paketus @CRAN
\begin{lstlisting}
a <- available.packages()
length(a)
\end{lstlisting}
\item Šiuo metu : 228140 paketai
\item taip pat galima ir iš github
\item install.packages("ggplot")
\item  install.packages(c("ggplot", "dplyr"))
\item iš R 
\item library(ggplot) čia nebereikia kabučių!
\item search() parodo visus įjungtus paketus
\end{itemize}
\end{frame}

%----------------------------------------------------------
\begin{frame}{Literatūra}
\begin{itemize}
\item The Art of Data Science, R.Peng, E.Matsui 1-3 skyriai
\end{itemize}
\end{frame}


%----------------------------------------------------------
\end{document}