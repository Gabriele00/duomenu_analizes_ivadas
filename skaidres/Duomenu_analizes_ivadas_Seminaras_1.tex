\documentclass[12pt,a4paper]{article}
\usepackage[utf8]{inputenc}
\usepackage[L7x]{fontenc}
\usepackage[lithuanian]{babel}
\usepackage{lmodern} %be neveikia bold, italic etc

\usepackage{amsmath}
\usepackage{amsfonts}
\usepackage{amssymb}

\author{Justas Mundeikis}
\title{1 Seminaras}
\begin{document}
\begin{enumerate}
\item Instaliuokite R
\item instaliuokite Rstudio
\item Instaliuokite Git
\item Susikurkite GitHub paskyrą
\item Naudodamiesi GIT, susikurkite folderį "duomenuanalizė" initializuokite ir commitinkite
\item Įrašykite savo GitHub paskyros nurodą į....
\item Lokaliai sukurkite folderį "ivadas", kuriame sukurkite failą readme.md
\item readme.md įrašykite :
Info
* Vardas Pavardė
* Studento numeris
* Kursas/Grupė
\item pushinkit pokyčius į GitHub
\item Forkinkit justas.mundeikis/...
\item Padaykite paleiskite Rstudio, instaluokite paketus "ggplot" ir "dplyr", paleiskite juos . Padaykite ekrano printscreen. Vaizda .png arba /jpeg formatu išsaugokite /duomenuanalizė/įvadas/
\end{enumerate}
\end{document}