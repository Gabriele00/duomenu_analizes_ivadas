\documentclass[11pt,xcolor=table]{beamer}
\usetheme{CambridgeUS}
\usecolortheme{dolphin}

\usepackage[utf8]{inputenc}
\usepackage[L7x]{fontenc}
\usepackage[lithuanian]{babel}
\usepackage{lmodern} %be neveikia bold, italic etc


\usepackage{amsmath}
\usepackage{amsfonts}
\usepackage{amssymb}

\usepackage{graphicx}
\graphicspath{{./figures/}}

\usepackage{adjustbox}
\usepackage{bm}
\usepackage{subcaption}

\usepackage{multirow} % for multirow tables
\setbeamertemplate{caption}[numbered] % for numbering captions


\usepackage{listings}

\usepackage{tikz, pgfplots}
\usetikzlibrary{patterns,decorations.pathreplacing}


\author{Justas Mundeikis}
\title{Duomenų analizės įvadas}
%\setbeamercovered{transparent} 
%\setbeamertemplate{navigation symbols}{} 
%\logo{} 
%\institute{} 
%\date{} 
%\subject{} 



%----------------------------------------------------------
\begin{document}
%----------------------------------------------------------
\begin{frame}
\titlepage
\end{frame}
%----------------------------------------------------------
\begin{frame}{Šio kurso turinys:}
\begin{enumerate}
\item Įvadas į duomenų analizę 
\item Įvadas į R, RStudio, Git ir GitHub
\item R programavimo pagrindai
\item Tiriamoji duomenų analizė (angl.: exploratory data analysis)
\item Atkartojami tyrimai (Latex, RMarkdown)
\end{enumerate}
\end{frame}
%----------------------------------------------------------
\begin{frame}{Kurso tikslas}

{\Large In God we trust, all others bring data}\\ 
- W. Edwards Deming 

\begin{itemize}
\item Suteikti pagrindines reikalingas kompetencijas darbui su duomenimis
\item Šis kursas labiau taikomasis, todėl nėra aiškios takoskyros tarp paskaitų ir seminarų
\item Kursas savaime yra "lengvas", tačiau reikalauja daug praktinio darbo pastangų
\item Dirbama tik su R, RStudio, Git ir Github
\item Neliesime Matlab, Python. Galbūt prisiliesime prie Eviews
\item Šis yra pirmas šio kurso "blynas"
\item Kiek laiko reikia \textbf{savistudijoms}?
\item Vietiniai PC vs. nuosavi notebook'ai
\end{itemize}
\end{frame}
%----------------------------------------------------------
\begin{frame}{Kurso vertinimo strategija}

\begin{itemize}
\item Savaitiniai namų darbai - 30\%
\item Neanonsuoti testai - 30\%
\item Baigiamasis egzaminas - 40\%
\end{itemize}
Savaitinių namų darbų tikslas parodyti studentams kokio pobūdžio uždavinius studentai turi gebėti savarankiškai spręsti.\\
Neanonsuoti trumpi testai skirti užtikrinti, jog studentai nuolatos skirtų deramą dėmesį ir laiku įdėtų reikalingas pastangas studijoms.\\
Egzamino metu bus tikrinamos tiek teorijos tiek įgytos praktinės žinios 
\end{frame}
%----------------------------------------------------------



























%----------------------------------------------------------
\end{document}